\documentclass[../main.tex]{subfiles}
\graphicspath{\subfix{../images}}{\subfix{../figures}}

\begin{document}
\section{集合}
\begin{introduction}
    \item 集合的概念~\ref{def:naive-set}
    \item 子集~\ref{def:subset}
    \item 真子集~\ref{def:proper-subset}
    \item 空集~\ref{def:empty-set}
\end{introduction}

集合概念的发展历史悠久
\improvement{关于集合的发展史,有待完善,目的:能够大致提现其进程,可以让人看到集合论的大致全貌}

\subsection{集合的概念}

\begin{definition}{集合(非定义的定义)(set)}{naive-set}
    \textcolor{main}{集合}(简称集)是指具有某种特定性质的物质或者是抽象对象组成的整体,在数学上这些物质或者抽象对象通常就是我们的研究对象,一般地,我们把研究对象统称为\textcolor{main}{元素}(element)或成员.
\end{definition}

\subsection{集合见的基本关系}
\begin{definition}{子集(subset)}{subset}
   对于两个集合 $A$, $B$, 如果集合 $A$ 中任意一个元素都是集合 $B$ 中的元素, 就称集合 $A$ 是集合 $B$ 的\textcolor{main}{子集},记作 $A \subseteq B$ (或 ), 读作 "$A$ 包含与 $B$" (或 "$B$ 包含 $A$"). \\
   $A \subseteq B :\Leftrightarrow (\forall x \in A: x \in B)$
\end{definition}

\begin{definition}{真子集(proper subset)}{proper-subset}
   如果集合 $A \subseteq B$, 但存在元素 $x \in B$,且 $x \notin A$, 就称集合 $A$ 是集合 $B$ 的\textcolor{main}{真子集},记作 $A \subsetneqq B$ (或), 读作 "$A$ 真包含于 $B$" (或 "$B$ 真包含 $A$").\\
   $A \subsetneqq B :\Leftrightarrow (A \subseteq B) \land (\exists x_0 \in B: x_0 \notin A)$
\end{definition}

\begin{definition}{空集}{empty-set}
    不含任何元素的集合叫做\textcolor{main}{空集}, 记作 $\varnothing$, 并规定\textbf{空集是任何集合的子集}.
\end{definition}

\begin{property}\label{property:set-prop}
    对于任意的集合 $A, B, C$,有:
    \begin{enumerate}
        \item $A \subseteq A$;
        \item 若 $A \subseteq B$, 且 $B \subseteq C$, 则 $A \subseteq C$.
    \end{enumerate}
\end{property}

% 集合元素的个数(Cardinality of the set)
\improvement[inline]{集合元素的个数(Cardinality of the set)相关知识有待完善}


\subsection{集合的基本运算}

\begin{definition}{并集 (union set)}{union-set}
    设有集合 $A$, $B$, $C$, 若集合 $C$ 是由所有属于集合 $A$ \textcolor{second}{或}属于集合 $B$ 的元素组成的, 那么我们称集合 $C$ 是集合 $A$ 与 $B$ 的\textcolor{main}{并集}, 记作 $A \cup B$, 读作 "$A \mbox{并} B". \\
    \begin{equation*}
        C = A \cup B = \{ x \mid x \in A, \mbox{或} x \in B \}
    \end{equation*}
\end{definition}

\begin{definition}{交集 (intersection set)}{intersection-set}
    设有集合 $A$, $B$, $C$, 若集合 $C$ 是由所有属于集合 $A$ \textcolor{second}{且}属于集合 $B$ 的元素组成的, 那么我们称集合 $C$ 是集合 $A$ 与 $B$ 的\textcolor{main}{交集}, 记作 $A \cap B$, 读作 "$A \mbox{交} B". \\
    \begin{equation*}
        C = A \cap B = \{ x \mid x \in A, \mbox{且} x \in B \}
    \end{equation*}
\end{definition}

\begin{definition}{全集 (universe set)}{universe-set}
    
\end{definition}

\section{常用逻辑用语}

\subsection{充分条件与必要条件}

\subsection{全称量词与存在量词}

\section{等式与不等式}

\subsection{等式与不等式的性质}

\subsection{基本不等式}

\subsection{二次函数与一元二次方程、不等式}

\end{document}